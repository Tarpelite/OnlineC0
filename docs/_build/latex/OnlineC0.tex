%% Generated by Sphinx.
\def\sphinxdocclass{report}
\documentclass[letterpaper,10pt,english]{sphinxmanual}
\ifdefined\pdfpxdimen
   \let\sphinxpxdimen\pdfpxdimen\else\newdimen\sphinxpxdimen
\fi \sphinxpxdimen=.75bp\relax

\PassOptionsToPackage{warn}{textcomp}
\usepackage[utf8]{inputenc}
\ifdefined\DeclareUnicodeCharacter
 \ifdefined\DeclareUnicodeCharacterAsOptional
  \DeclareUnicodeCharacter{"00A0}{\nobreakspace}
  \DeclareUnicodeCharacter{"2500}{\sphinxunichar{2500}}
  \DeclareUnicodeCharacter{"2502}{\sphinxunichar{2502}}
  \DeclareUnicodeCharacter{"2514}{\sphinxunichar{2514}}
  \DeclareUnicodeCharacter{"251C}{\sphinxunichar{251C}}
  \DeclareUnicodeCharacter{"2572}{\textbackslash}
 \else
  \DeclareUnicodeCharacter{00A0}{\nobreakspace}
  \DeclareUnicodeCharacter{2500}{\sphinxunichar{2500}}
  \DeclareUnicodeCharacter{2502}{\sphinxunichar{2502}}
  \DeclareUnicodeCharacter{2514}{\sphinxunichar{2514}}
  \DeclareUnicodeCharacter{251C}{\sphinxunichar{251C}}
  \DeclareUnicodeCharacter{2572}{\textbackslash}
 \fi
\fi
\usepackage{cmap}
\usepackage[T1]{fontenc}
\usepackage{amsmath,amssymb,amstext}
\usepackage{babel}
\usepackage{times}
\usepackage[Sonny]{fncychap}
\ChNameVar{\Large\normalfont\sffamily}
\ChTitleVar{\Large\normalfont\sffamily}
\usepackage{sphinx}

\usepackage{geometry}

% Include hyperref last.
\usepackage{hyperref}
% Fix anchor placement for figures with captions.
\usepackage{hypcap}% it must be loaded after hyperref.
% Set up styles of URL: it should be placed after hyperref.
\urlstyle{same}
\addto\captionsenglish{\renewcommand{\contentsname}{Contents:}}

\addto\captionsenglish{\renewcommand{\figurename}{图}}
\addto\captionsenglish{\renewcommand{\tablename}{表}}
\addto\captionsenglish{\renewcommand{\literalblockname}{列表}}

\addto\captionsenglish{\renewcommand{\literalblockcontinuedname}{续上页}}
\addto\captionsenglish{\renewcommand{\literalblockcontinuesname}{continues on next page}}

\addto\extrasenglish{\def\pageautorefname{页}}

\setcounter{tocdepth}{2}



\title{OnlineC0 Documentation}
\date{2018 年 12 月 10 日}
\release{}
\author{Tarpelite}
\newcommand{\sphinxlogo}{\vbox{}}
\renewcommand{\releasename}{}
\makeindex

\begin{document}

\maketitle
\sphinxtableofcontents
\phantomsection\label{\detokenize{index::doc}}



\chapter{内容}
\label{\detokenize{index:id1}}\label{\detokenize{index:onlinec0}}

\section{C0 编译器进度}
\label{\detokenize{intro:c0}}\label{\detokenize{intro::doc}}

\subsection{词法分析}
\label{\detokenize{intro:id1}}
首先,选择一符一类的方式来分析C0语法,得到分析结果如下:

\sphinxstyleemphasis{终结符Vt(除关键字):}
\begin{enumerate}
\item {} 
‘+’

\item {} 
‘-‘

\item {} 
‘*’

\item {} 
/

\item {} 
\textless{}

\item {} 
\textless{}=

\item {} 
\textgreater{}

\item {} 
\textgreater{}=

\item {} 
!=

\item {} 
==

\item {} 
\_azAZ

\item {} 
0-9

\item {} 
(

\item {} 
)

\item {} 
\{

\item {} 
\}

\item {} 
,

\item {} 
;

\item {} 
=

\item {} 
.

\item {} 
:

\end{enumerate}

\sphinxstyleemphasis{关键字:}
\begin{enumerate}
\item {} 
const

\item {} 
int

\item {} 
void

\item {} 
if

\item {} 
else

\item {} 
while

\item {} 
main

\item {} 
return

\item {} 
printf

\item {} 
scanf

\end{enumerate}

\sphinxstyleemphasis{标识符}

1.变量名
2.数组名
3.函数名

\sphinxstyleemphasis{常数:}

1.整数
2.浮点数
3.字符串

\sphinxstyleemphasis{分界符:}
\begin{enumerate}
\item {} 
‘+’

\item {} 
‘-‘

\item {} 
‘*’

\item {} 
/

\item {} 
\textless{}

\item {} 
\textless{}=

\item {} 
\textgreater{}

\item {} 
\textgreater{}=

\item {} 
!=

\item {} 
==

\item {} 
(

\item {} 
)

\item {} 
\{

\item {} 
\}

\item {} 
,

\item {} 
;

\item {} 
=

\item {} 
.

\item {} 
:

\item {} 
“

\item {} 
“

\end{enumerate}

\sphinxstyleemphasis{双目分界符:}
\begin{enumerate}
\item {} 
\textless{}=

\item {} 
\textgreater{}=

\item {} 
!=

\item {} 
==

\end{enumerate}


\section{前端交互页面views}
\label{\detokenize{lexer/views::doc}}\label{\detokenize{lexer/views:module-lexer.views}}\label{\detokenize{lexer/views:views}}\index{lexer.views (模块)}\index{complie() (在 lexer.views 模块中)}

\begin{fulllineitems}
\phantomsection\label{\detokenize{lexer/views:lexer.views.complie}}\pysiglinewithargsret{\sphinxcode{\sphinxupquote{lexer.views.}}\sphinxbfcode{\sphinxupquote{complie}}}{\emph{request}}{}
对POST请求进行处理,去掉表单中多余的字符后交由词法分析器,
再将渲染的结果返回给客户端

\end{fulllineitems}



\section{词法分析器Lexer}
\label{\detokenize{lexer/lexer::doc}}\label{\detokenize{lexer/lexer:lexer}}\index{C0lexer (lexer.lexer 中的类)}

\begin{fulllineitems}
\phantomsection\label{\detokenize{lexer/lexer:lexer.lexer.C0lexer}}\pysiglinewithargsret{\sphinxbfcode{\sphinxupquote{class }}\sphinxcode{\sphinxupquote{lexer.lexer.}}\sphinxbfcode{\sphinxupquote{C0lexer}}}{\emph{INPUT\_FILE\_NAME: str}}{}
多功能词法分析器
\index{getsym() (lexer.lexer.C0lexer 方法)}

\begin{fulllineitems}
\phantomsection\label{\detokenize{lexer/lexer:lexer.lexer.C0lexer.getsym}}\pysiglinewithargsret{\sphinxbfcode{\sphinxupquote{getsym}}}{}{}
分析下一个TOKEN

\end{fulllineitems}

\index{clearToken() (lexer.lexer.C0lexer 方法)}

\begin{fulllineitems}
\phantomsection\label{\detokenize{lexer/lexer:lexer.lexer.C0lexer.clearToken}}\pysiglinewithargsret{\sphinxbfcode{\sphinxupquote{clearToken}}}{}{}
清空目前已经读取的字符缓存

\end{fulllineitems}

\index{curChar() (lexer.lexer.C0lexer 方法)}

\begin{fulllineitems}
\phantomsection\label{\detokenize{lexer/lexer:lexer.lexer.C0lexer.curChar}}\pysiglinewithargsret{\sphinxbfcode{\sphinxupquote{curChar}}}{}{}
返回当前指针指向的字符

\end{fulllineitems}

\index{isDigit() (lexer.lexer.C0lexer 方法)}

\begin{fulllineitems}
\phantomsection\label{\detokenize{lexer/lexer:lexer.lexer.C0lexer.isDigit}}\pysiglinewithargsret{\sphinxbfcode{\sphinxupquote{isDigit}}}{}{}
判断当前字符是否为数字

\end{fulllineitems}

\index{isLetter() (lexer.lexer.C0lexer 方法)}

\begin{fulllineitems}
\phantomsection\label{\detokenize{lexer/lexer:lexer.lexer.C0lexer.isLetter}}\pysiglinewithargsret{\sphinxbfcode{\sphinxupquote{isLetter}}}{}{}
判断当前字符为字母

\end{fulllineitems}

\index{getchar() (lexer.lexer.C0lexer 方法)}

\begin{fulllineitems}
\phantomsection\label{\detokenize{lexer/lexer:lexer.lexer.C0lexer.getchar}}\pysiglinewithargsret{\sphinxbfcode{\sphinxupquote{getchar}}}{}{}
将当前字符加入TOKEN末尾
文件指针+1

\end{fulllineitems}

\index{tobin() (lexer.lexer.C0lexer 方法)}

\begin{fulllineitems}
\phantomsection\label{\detokenize{lexer/lexer:lexer.lexer.C0lexer.tobin}}\pysiglinewithargsret{\sphinxbfcode{\sphinxupquote{tobin}}}{}{}
十进制转二进制

\end{fulllineitems}

\index{error() (lexer.lexer.C0lexer 方法)}

\begin{fulllineitems}
\phantomsection\label{\detokenize{lexer/lexer:lexer.lexer.C0lexer.error}}\pysiglinewithargsret{\sphinxbfcode{\sphinxupquote{error}}}{}{}
出错提示

\end{fulllineitems}

\index{word\_analyze() (lexer.lexer.C0lexer 方法)}

\begin{fulllineitems}
\phantomsection\label{\detokenize{lexer/lexer:lexer.lexer.C0lexer.word_analyze}}\pysiglinewithargsret{\sphinxbfcode{\sphinxupquote{word\_analyze}}}{}{}
自动词法分析至出错或到文件末尾

\end{fulllineitems}

\index{web\_word\_analyze() (lexer.lexer.C0lexer 方法)}

\begin{fulllineitems}
\phantomsection\label{\detokenize{lexer/lexer:lexer.lexer.C0lexer.web_word_analyze}}\pysiglinewithargsret{\sphinxbfcode{\sphinxupquote{web\_word\_analyze}}}{}{}
用于服务器的词法分析接口

\end{fulllineitems}

\index{output() (lexer.lexer.C0lexer 方法)}

\begin{fulllineitems}
\phantomsection\label{\detokenize{lexer/lexer:lexer.lexer.C0lexer.output}}\pysiglinewithargsret{\sphinxbfcode{\sphinxupquote{output}}}{}{}
输出词法分析结果文件

\end{fulllineitems}


\end{fulllineitems}



\section{算法优先文法分析器OPGA}
\label{\detokenize{OPG/OPGA::doc}}\label{\detokenize{OPG/OPGA:opga}}

\section{语法分析}
\label{\detokenize{grammar_analysis/gramma::doc}}\label{\detokenize{grammar_analysis/gramma:id1}}\index{special\_lexer (grammar\_analysis.utils 中的类)}

\begin{fulllineitems}
\phantomsection\label{\detokenize{grammar_analysis/gramma:grammar_analysis.utils.special_lexer}}\pysiglinewithargsret{\sphinxbfcode{\sphinxupquote{class }}\sphinxcode{\sphinxupquote{grammar\_analysis.utils.}}\sphinxbfcode{\sphinxupquote{special\_lexer}}}{\emph{INPUT\_FILE\_NAME='web'}}{}
词法分析器,是C0lexer的子类
\index{clearToken() (grammar\_analysis.utils.special\_lexer 方法)}

\begin{fulllineitems}
\phantomsection\label{\detokenize{grammar_analysis/gramma:grammar_analysis.utils.special_lexer.clearToken}}\pysiglinewithargsret{\sphinxbfcode{\sphinxupquote{clearToken}}}{}{}
清空目前已经读取的字符缓存

\end{fulllineitems}

\index{curChar() (grammar\_analysis.utils.special\_lexer 方法)}

\begin{fulllineitems}
\phantomsection\label{\detokenize{grammar_analysis/gramma:grammar_analysis.utils.special_lexer.curChar}}\pysiglinewithargsret{\sphinxbfcode{\sphinxupquote{curChar}}}{}{}
返回当前指针指向的字符

\end{fulllineitems}

\index{error() (grammar\_analysis.utils.special\_lexer 方法)}

\begin{fulllineitems}
\phantomsection\label{\detokenize{grammar_analysis/gramma:grammar_analysis.utils.special_lexer.error}}\pysiglinewithargsret{\sphinxbfcode{\sphinxupquote{error}}}{}{}
出错提示

\end{fulllineitems}

\index{error\_report() (grammar\_analysis.utils.special\_lexer 方法)}

\begin{fulllineitems}
\phantomsection\label{\detokenize{grammar_analysis/gramma:grammar_analysis.utils.special_lexer.error_report}}\pysiglinewithargsret{\sphinxbfcode{\sphinxupquote{error\_report}}}{}{}
错误分析报告

\end{fulllineitems}

\index{getchar() (grammar\_analysis.utils.special\_lexer 方法)}

\begin{fulllineitems}
\phantomsection\label{\detokenize{grammar_analysis/gramma:grammar_analysis.utils.special_lexer.getchar}}\pysiglinewithargsret{\sphinxbfcode{\sphinxupquote{getchar}}}{}{}
将当前字符加入TOKEN末尾
文件指针+1

\end{fulllineitems}

\index{isDigit() (grammar\_analysis.utils.special\_lexer 方法)}

\begin{fulllineitems}
\phantomsection\label{\detokenize{grammar_analysis/gramma:grammar_analysis.utils.special_lexer.isDigit}}\pysiglinewithargsret{\sphinxbfcode{\sphinxupquote{isDigit}}}{}{}
判断当前字符是否为数字

\end{fulllineitems}

\index{isLetter() (grammar\_analysis.utils.special\_lexer 方法)}

\begin{fulllineitems}
\phantomsection\label{\detokenize{grammar_analysis/gramma:grammar_analysis.utils.special_lexer.isLetter}}\pysiglinewithargsret{\sphinxbfcode{\sphinxupquote{isLetter}}}{}{}
判断当前字符为字母

\end{fulllineitems}

\index{isNewLine() (grammar\_analysis.utils.special\_lexer 方法)}

\begin{fulllineitems}
\phantomsection\label{\detokenize{grammar_analysis/gramma:grammar_analysis.utils.special_lexer.isNewLine}}\pysiglinewithargsret{\sphinxbfcode{\sphinxupquote{isNewLine}}}{}{}
\end{fulllineitems}

\index{isSpace() (grammar\_analysis.utils.special\_lexer 方法)}

\begin{fulllineitems}
\phantomsection\label{\detokenize{grammar_analysis/gramma:grammar_analysis.utils.special_lexer.isSpace}}\pysiglinewithargsret{\sphinxbfcode{\sphinxupquote{isSpace}}}{}{}
判断当前字符是否为’/r’或者”/n”

\end{fulllineitems}

\index{output() (grammar\_analysis.utils.special\_lexer 方法)}

\begin{fulllineitems}
\phantomsection\label{\detokenize{grammar_analysis/gramma:grammar_analysis.utils.special_lexer.output}}\pysiglinewithargsret{\sphinxbfcode{\sphinxupquote{output}}}{}{}
输出词法分析结果文件

\end{fulllineitems}

\index{print\_result() (grammar\_analysis.utils.special\_lexer 方法)}

\begin{fulllineitems}
\phantomsection\label{\detokenize{grammar_analysis/gramma:grammar_analysis.utils.special_lexer.print_result}}\pysiglinewithargsret{\sphinxbfcode{\sphinxupquote{print\_result}}}{}{}
打印结果

\end{fulllineitems}

\index{read() (grammar\_analysis.utils.special\_lexer 方法)}

\begin{fulllineitems}
\phantomsection\label{\detokenize{grammar_analysis/gramma:grammar_analysis.utils.special_lexer.read}}\pysiglinewithargsret{\sphinxbfcode{\sphinxupquote{read}}}{}{}
如果以文件形式输入,调用该函数执行读操作

\end{fulllineitems}

\index{tobin() (grammar\_analysis.utils.special\_lexer 方法)}

\begin{fulllineitems}
\phantomsection\label{\detokenize{grammar_analysis/gramma:grammar_analysis.utils.special_lexer.tobin}}\pysiglinewithargsret{\sphinxbfcode{\sphinxupquote{tobin}}}{\emph{x: int}}{}
十进制转二进制

\end{fulllineitems}

\index{web\_input() (grammar\_analysis.utils.special\_lexer 方法)}

\begin{fulllineitems}
\phantomsection\label{\detokenize{grammar_analysis/gramma:grammar_analysis.utils.special_lexer.web_input}}\pysiglinewithargsret{\sphinxbfcode{\sphinxupquote{web\_input}}}{\emph{SOURCE\_TEXT}}{}
从表单输入文本

\end{fulllineitems}

\index{web\_reply() (grammar\_analysis.utils.special\_lexer 方法)}

\begin{fulllineitems}
\phantomsection\label{\detokenize{grammar_analysis/gramma:grammar_analysis.utils.special_lexer.web_reply}}\pysiglinewithargsret{\sphinxbfcode{\sphinxupquote{web\_reply}}}{}{}
用于服务器的分析应答接口

\end{fulllineitems}

\index{web\_word\_analyze() (grammar\_analysis.utils.special\_lexer 方法)}

\begin{fulllineitems}
\phantomsection\label{\detokenize{grammar_analysis/gramma:grammar_analysis.utils.special_lexer.web_word_analyze}}\pysiglinewithargsret{\sphinxbfcode{\sphinxupquote{web\_word\_analyze}}}{}{}
用于服务器的词法分析接口

\end{fulllineitems}

\index{word\_analyze() (grammar\_analysis.utils.special\_lexer 方法)}

\begin{fulllineitems}
\phantomsection\label{\detokenize{grammar_analysis/gramma:grammar_analysis.utils.special_lexer.word_analyze}}\pysiglinewithargsret{\sphinxbfcode{\sphinxupquote{word\_analyze}}}{}{}
自动词法分析至出错或到文件末尾

\end{fulllineitems}


\end{fulllineitems}



\chapter{正在施工中的网站(欢迎蹂躏)}
\label{\detokenize{index:id2}}
visit \sphinxhref{http://www.buaatech.top:8080/lexer/}{Tarpe酋长金坷垃C0编译器}


\renewcommand{\indexname}{Python 模块索引}
\begin{sphinxtheindex}
\def\bigletter#1{{\Large\sffamily#1}\nopagebreak\vspace{1mm}}
\bigletter{l}
\item {\sphinxstyleindexentry{lexer.views}}\sphinxstyleindexpageref{lexer/views:\detokenize{module-lexer.views}}
\end{sphinxtheindex}

\renewcommand{\indexname}{索引}
\printindex
\end{document}